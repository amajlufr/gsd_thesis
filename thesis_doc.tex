% Options for packages loaded elsewhere
\PassOptionsToPackage{unicode}{hyperref}
\PassOptionsToPackage{hyphens}{url}
\PassOptionsToPackage{dvipsnames,svgnames,x11names}{xcolor}
%
\documentclass[
  12pt,
]{report}

\usepackage{amsmath,amssymb}
\usepackage{iftex}
\ifPDFTeX
  \usepackage[T1]{fontenc}
  \usepackage[utf8]{inputenc}
  \usepackage{textcomp} % provide euro and other symbols
\else % if luatex or xetex
  \usepackage{unicode-math}
  \defaultfontfeatures{Scale=MatchLowercase}
  \defaultfontfeatures[\rmfamily]{Ligatures=TeX,Scale=1}
\fi
\usepackage{lmodern}
\ifPDFTeX\else  
    % xetex/luatex font selection
\fi
% Use upquote if available, for straight quotes in verbatim environments
\IfFileExists{upquote.sty}{\usepackage{upquote}}{}
\IfFileExists{microtype.sty}{% use microtype if available
  \usepackage[]{microtype}
  \UseMicrotypeSet[protrusion]{basicmath} % disable protrusion for tt fonts
}{}
\makeatletter
\@ifundefined{KOMAClassName}{% if non-KOMA class
  \IfFileExists{parskip.sty}{%
    \usepackage{parskip}
  }{% else
    \setlength{\parindent}{0pt}
    \setlength{\parskip}{6pt plus 2pt minus 1pt}}
}{% if KOMA class
  \KOMAoptions{parskip=half}}
\makeatother
\usepackage{xcolor}
\setlength{\emergencystretch}{3em} % prevent overfull lines
\setcounter{secnumdepth}{-\maxdimen} % remove section numbering
% Make \paragraph and \subparagraph free-standing
\makeatletter
\ifx\paragraph\undefined\else
  \let\oldparagraph\paragraph
  \renewcommand{\paragraph}{
    \@ifstar
      \xxxParagraphStar
      \xxxParagraphNoStar
  }
  \newcommand{\xxxParagraphStar}[1]{\oldparagraph*{#1}\mbox{}}
  \newcommand{\xxxParagraphNoStar}[1]{\oldparagraph{#1}\mbox{}}
\fi
\ifx\subparagraph\undefined\else
  \let\oldsubparagraph\subparagraph
  \renewcommand{\subparagraph}{
    \@ifstar
      \xxxSubParagraphStar
      \xxxSubParagraphNoStar
  }
  \newcommand{\xxxSubParagraphStar}[1]{\oldsubparagraph*{#1}\mbox{}}
  \newcommand{\xxxSubParagraphNoStar}[1]{\oldsubparagraph{#1}\mbox{}}
\fi
\makeatother


\providecommand{\tightlist}{%
  \setlength{\itemsep}{0pt}\setlength{\parskip}{0pt}}\usepackage{longtable,booktabs,array}
\usepackage{calc} % for calculating minipage widths
% Correct order of tables after \paragraph or \subparagraph
\usepackage{etoolbox}
\makeatletter
\patchcmd\longtable{\par}{\if@noskipsec\mbox{}\fi\par}{}{}
\makeatother
% Allow footnotes in longtable head/foot
\IfFileExists{footnotehyper.sty}{\usepackage{footnotehyper}}{\usepackage{footnote}}
\makesavenoteenv{longtable}
\usepackage{graphicx}
\makeatletter
\def\maxwidth{\ifdim\Gin@nat@width>\linewidth\linewidth\else\Gin@nat@width\fi}
\def\maxheight{\ifdim\Gin@nat@height>\textheight\textheight\else\Gin@nat@height\fi}
\makeatother
% Scale images if necessary, so that they will not overflow the page
% margins by default, and it is still possible to overwrite the defaults
% using explicit options in \includegraphics[width, height, ...]{}
\setkeys{Gin}{width=\maxwidth,height=\maxheight,keepaspectratio}
% Set default figure placement to htbp
\makeatletter
\def\fps@figure{htbp}
\makeatother

\usepackage{geometry}
\geometry{
  top=1in,
  bottom=1in,
  left=1in,
  right=1in}
\usepackage{setspace}
\setstretch{1}
\usepackage{fontspec}
\setmainfont{Cormorant Garamond}
\setsansfont{Fira Sans}
\setmonofont{Fira Mono}
\makeatletter
\@ifpackageloaded{caption}{}{\usepackage{caption}}
\AtBeginDocument{%
\ifdefined\contentsname
  \renewcommand*\contentsname{Table of contents}
\else
  \newcommand\contentsname{Table of contents}
\fi
\ifdefined\listfigurename
  \renewcommand*\listfigurename{List of Figures}
\else
  \newcommand\listfigurename{List of Figures}
\fi
\ifdefined\listtablename
  \renewcommand*\listtablename{List of Tables}
\else
  \newcommand\listtablename{List of Tables}
\fi
\ifdefined\figurename
  \renewcommand*\figurename{Figure}
\else
  \newcommand\figurename{Figure}
\fi
\ifdefined\tablename
  \renewcommand*\tablename{Table}
\else
  \newcommand\tablename{Table}
\fi
}
\@ifpackageloaded{float}{}{\usepackage{float}}
\floatstyle{ruled}
\@ifundefined{c@chapter}{\newfloat{codelisting}{h}{lop}}{\newfloat{codelisting}{h}{lop}[chapter]}
\floatname{codelisting}{Listing}
\newcommand*\listoflistings{\listof{codelisting}{List of Listings}}
\makeatother
\makeatletter
\makeatother
\makeatletter
\@ifpackageloaded{caption}{}{\usepackage{caption}}
\@ifpackageloaded{subcaption}{}{\usepackage{subcaption}}
\makeatother

\ifLuaTeX
  \usepackage{selnolig}  % disable illegal ligatures
\fi
\usepackage{bookmark}

\IfFileExists{xurl.sty}{\usepackage{xurl}}{} % add URL line breaks if available
\urlstyle{same} % disable monospaced font for URLs
\hypersetup{
  colorlinks=true,
  linkcolor={blue},
  filecolor={Maroon},
  citecolor={Blue},
  urlcolor={Blue},
  pdfcreator={LaTeX via pandoc}}


\author{}
\date{}

\begin{document}


\begin{center}

\thispagestyle{empty}

\vspace*{2in}

\Huge \textbf{ACCESSIBILITY AND REAL ESTATE PRICES IN MEXICO CITY: A HEDONIC APPROACH}

\vspace*{2in}

\Large \textbf{ABRAHAM HAIM MAJLUF RIZO}

\vspace*{0.25in}

HARVARD UNIVERSITY GRADUATE SCHOOL OF DESIGN\\[0em]
MASTER'S IN URBAN PLANNING THESIS

\vspace*{0.25in}

SPRING 2025

\end{center}

\chapter{INTRODUCTION}\label{introduction}

XXX

\newpage

\chapter{LITERATURE REVIEW}\label{literature-review}

\section{ACCESSIBILITY AS A CONCEPT}\label{accessibility-as-a-concept}

Accessibility has been a core topic in urban planning literature since
the mid-20th century (Hansen, 1959). Commonly defined as the ease of
reaching desired destinations, accessibility reflects both the mobility
afforded by transportation networks and the spatial distribution of
opportunities---such as jobs, services, or amenities---along with the
time, cost, and effort associated with travel (Boisjoly \& El-Geneidy,
2017; Levinson \& Wu, 2020).

Its significance lies in its influence on the economic value of urban
land, shaping the intensity and type of development that occurs across
different locations. Accessibility disparities are deeply intertwined
with broader patterns of spatial and social inequality, as variations in
access---often shaped by demographic and socioeconomic
factors---directly impact quality of life and underscore the need for
more equitable urban policy frameworks (Wachs \& Kumagai, 1973).

The concept has demonstrated relevance across a wide range of urban
processes, including commuting behaviors, employment access, service
provision, real estate valuation and development, environmental
sustainability, and social inclusion. Its centrality in shaping urban
form and function continues to make accessibility a foundational lens
through which urban dynamics are studied and interpreted (El-Geneidy \&
Levinson, 2022; Levinson \& Wu, 2020; Cervero, Rood, \& Appleyard, 1999;
Atuesta et al., 2018).

\section{ACCESSIBILITY MEASUREMENTS}\label{accessibility-measurements}

Accessibility measures vary widely in their conceptual foundations and
methodological approaches. As outlined by Siddiq and Taylor (2021), one
of the most straightforward approaches is the cumulative opportunities
measure, which quantifies the number of destinations---typically jobs or
services---reachable within a defined time or distance threshold. While
valued for their simplicity and ease of communication, these measures
assume all destinations offer equal utility, failing to account for
variations in quality or desirability. To address this limitation,
gravity-based measures incorporate a distance decay function that gives
greater weight to closer destinations, reflecting the reduced likelihood
of traveling to more distant locations. These models integrate the
generalized cost of travel---such as time, distance, and monetary
expense---offering a more nuanced representation of accessibility.
Utility-based measures take this further by shifting the analytical
focus from destinations to individuals or households. They incorporate
factors such as user preferences, socioeconomic attributes, and
perceptions of destination attractiveness, aiming to estimate the
expected maximum utility derived from a set of travel alternatives. This
allows for a more personalized and behaviorally grounded understanding
of accessibility. Finally, constraints-based or space--time measures
introduce a temporal dimension by acknowledging the scheduling and
sequencing of daily activities. These approaches recognize that
accessibility is not only spatial but also shaped by individuals'
available time windows and activity patterns.

In summary, accessibility measures span a spectrum from simple to
theoretically robust. Cumulative opportunity measures are easier to
compute and communicate but lack nuance, while more complex models
capture behavioral and temporal dimensions at the cost of increased data
and analytical demands (Siddiq \& Taylor, 2021). Despite the growing
sophistication of accessibility research, its practical application in
planning remains limited, often due to gaps between theoretical
frameworks and real-world constraints (Boisjoly \& El-Geneidy, 2017).

In addition to the conceptual variations in accessibility modeling,
differences in how impedance is specified can significantly affect
measurement outcomes, especially within the context of hedonic price
models. Impedance refers to the resistance or effort required to reach a
destination and is commonly categorized into five types: Euclidean
distance, network distance, travel time, cost, and zone-based measures
(Heyman et al., 2018). Euclidean distance represents straight-line
proximity and offers a basic, objective baseline. Network distance adds
realism by accounting for the actual traversable paths through a
transport network. Travel time introduces temporal dynamics, considering
congestion or average speed variations, while cost-based impedance
captures monetary expenses associated with travel. Zone-based measures
rely on spatial aggregations, such as administrative boundaries, and are
often used when data is limited. However, these zonal aggregations are
especially vulnerable to the Modifiable Areal Unit Problem (MAUP)---a
spatial bias that can distort results depending on the size and
configuration of the spatial units used. MAUP comprises two components:
the scale effect, where statistical outcomes shift depending on the
number or size of areal units, and the zoning effect, where boundary
definitions alone can lead to different conclusions. These issues
underscore the importance of using disaggregated data or carefully
chosen spatial units to preserve local variation and ensure alignment
with perceived accessibility (Heyman et al., 2018).

Striking a balance between simplicity and complexity is crucial.
Introducing stakeholders to basic cumulative measures may serve as a
useful entry point, while simultaneously advocating for the added
insights of distance-sensitive or utility-based models. As El-Geneidy
and Levinson (2022) argue, the challenge lies in translating theoretical
advancements into tools that are both actionable and accessible for
planning practice.

\section{ACCESSIBILITY IN THE HEDONIC PRICE MODEL
FRAMEWORK}\label{accessibility-in-the-hedonic-price-model-framework}

Hedonic pricing models (HPMs) provide a foundational framework for
estimating the implicit value of individual product characteristics by
analyzing observed prices and the attributes of differentiated goods.
Rooted in Rosen's (1974) seminal work, this approach views goods as
bundles of characteristics, aligning with theories of spatial
equilibrium and equalizing differences. In the real estate context, this
framework is especially relevant, as housing markets are inherently
heterogeneous---each property is unique in its physical features,
location, and historical usage, making precise valuation inherently
complex.

Unlike goods traded in efficient markets, real estate lacks product
homogeneity, frequent transactions, and transparent pricing. Properties
are not priced by individual attributes but as composite packages,
creating what Rosen (1974) describes as implicit markets. Buyers and
sellers negotiate based on subjective valuations of inseparable
traits---such as location, size, amenities, and architectural
style---rather than standardized units, which complicates pricing and
contributes to overall market inefficiency (Evans, 1995).

Despite these challenges, hedonic pricing models remain the dominant
method for estimating residential property values, breaking down a
home's sale price into the marginal contributions of its attributes.
Since the mid-2000s, the literature on HPMs has expanded significantly,
addressing key market disruptions (e.g., the Great Recession),
integrating new thematic concerns such as climate vulnerability,
financing structures, and neighborhood dynamics, and employing
increasingly advanced methodologies. Recent approaches include
instrumental variable techniques (e.g., 2SLS), hazard models, quantile
regression, and machine learning tools such as random
forests---reflecting the complexity and richness of contemporary housing
data (Khoshnoud et al., 2023).

Accessibility is a critical yet variably treated component in HPMs. Its
influence on housing prices is well-documented, but its measurement
varies significantly across studies, affecting both theoretical
soundness and empirical reliability. As Heyman et al.~(2018) note,
accessibility can be conceptualized using several frameworks---including
spatial separation, cumulative opportunity, gravitational potential,
utility-based access, and time-space constraints---each offering a
distinct lens. The choice of measure can significantly shape outcomes,
particularly given that buyers perceive and value accessibility
differently depending on local context and individual preferences.

Different modeling approaches have also emerged for constructing housing
price indices, each with strengths and limitations. Repeat-sales indices
track changes in property values by comparing repeat transactions of the
same property. While effective in controlling for unobservable
characteristics, these models face sample selection biases and assume
stability in property attributes over time. In contrast,
hedonic-regression-based indices allow for greater control over
attribute-specific changes but are sensitive to omitted variables and
model specification choices. As Wallace and Meese (1997) emphasize,
hedonic indices may be more reliable for assessing local market
behavior, while simpler metrics such as median sales prices remain
useful for tracking broader housing market cycles.

The era of big data is transforming real estate valuation, introducing
new data sources---from online listings and remote sensing to IoT
devices---and enabling models that can integrate real-time,
multidimensional inputs. Machine learning models, including artificial
neural networks, support vector machines, deep learning, and
gradient-boosting algorithms such as XGBoost, offer strong predictive
performance and adaptability to complex data structures. However, these
models introduce new challenges around interpretability, particularly in
policy contexts where transparency and accountability are critical (Wei
et al., 2022). In modeling housing prices, accessibility specifications
have traditionally relied on proximity to central business districts
(CBDs), major roads, or public transit stations. However, recent machine
learning models have revealed that street network connectivity, or local
street closeness, may be a more stable and informative predictor of home
values than simple CBD proximity (Liu et al., 2024). This insight
challenges conventional assumptions in urban economic models and
highlights the evolving role of urban form in shaping property values.

Ultimately, the field continues to evolve toward more nuanced,
behaviorally grounded, and technologically enabled valuation methods.
Yet, the integration of accessibility remains a central concern, as its
conceptualization and measurement directly influence valuation accuracy.
Bridging this gap---between methodological sophistication and practical
applicability---remains a key task for both researchers and
practitioners seeking to better understand and shape urban real estate
dynamics.

\subsection{CURRENT EVIDENCE ON MEXICO
CITY}\label{current-evidence-on-mexico-city}

In the specific case of Mexico City, only one study performed by Atuesta
et al.~(2018) was found. Here, the authors use geographically referenced
data on new housing developments to estimate how households value
accessibility to employment and transport infrastructure.

Their results indicate that proximity to employment subcenters increases
housing values by 1--3\%, suggesting that access to formal job markets
is capitalized into real estate prices. In contrast, proximity to the
Central Business District (CBD) is considered a disamenity, reflecting
the declining relevance of the historic core and the ongoing
decentralization of employment in the city. The effects of proximity to
Metro stations are non-linear: while homes located directly adjacent to
stations may be discounted due to congestion, informal commerce, and
insecurity, a moderate distance from Metro infrastructure is associated
with higher property values.

These patterns are shaped by sharp socioeconomic divides. High-income
households, who are more likely to own private vehicles, benefit more
from accessibility and enjoy shorter commutes, while low-income
households tend to live farther from job centers, often relying on
informal employment and transport modes. This spatial mismatch
reinforces patterns of segregation and limits access to economic
opportunity.

The authors argue that these challenges cannot be addressed through
infrastructure investment alone. Instead, they emphasize the importance
of integrated land-use and transport planning---such as transit-oriented
development and the revitalization of urban cores---to make public
transit systems more attractive and equitable across income groups.

\newpage

\chapter{METHODOLOGY}\label{methodology}

This study adopts a theoretical framework that explores the relationship
between real estate prices and three key dimensions: accessibility,
dwelling quality, and neighborhood characteristics. The following
sections detail the procedures for data collection and processing
related to each of these components.

\section{REAL ESTATE LISTINGS}\label{real-estate-listings}

To compile a structured dataset of current real estate listings in
Mexico City, a custom R script was developed to extract information from
paginated JSON endpoints provided by a major online real estate broker.
For each listing, the script captured a range of attributes, including
listing identifiers and timestamps; property characteristics such as
type, total and built surface area, number of bedrooms and bathrooms,
and the presence of amenities like parking; as well as pricing
information, including listed price and currency. Geographic details
were also collected, covering street address, neighborhood, and
geolocation coordinates.

The data collection script was executed 16 times over a three-month
period, from January to April 2025, with runs spaced at regular
intervals to capture temporal variation in the real estate market. This
repeated extraction enabled the construction of a longitudinal dataset,
after which a cleaning and filtering procedure was applied. Duplicate
entries were removed by retaining only the most recent record for each
property, based on the retrieval date. The dataset was then filtered to
include only properties listed for sale within the administrative
boundaries of Mexico City and restricted to residential property types.
Listing prices were standardized by converting all values to U.S.
dollars using a fixed exchange rate of 20 MXN per USD for entries
originally priced in Mexican pesos. The final dataset contains 1,560
observations.

\section{ACCESSIBILITY INDEX}\label{accessibility-index}

While real estate listings served as the origin points in the
accessibility analysis, destinations were drawn from the most recent
National Directory of Economic Units (DENUE, 2024). This dataset
provides granular, georreferenced information on individual economic
establishments, including their 6-digit NAICS classification and
corresponding employment size bands. These employment bands were
converted into approximate numeric estimates using the midpoint of each
range, with a tailored adjustment for the highest category to better
reflect potential outliers. Establishments located within both Mexico
City and adjacent municipalities in neighboring states were included to
ensure spatial continuity, particularly for listings situated near the
administrative boundaries of the city. In parallel, public space data
was sourced from Mexico City open data portal.

To support the routing engine required for travel time estimation, road
network data were sourced from OpenStreetMap for both Mexico City and
the State of Mexico. While state geometries were merged within R, the
final consolidated routable network in Protocolbuffer Binary Format
(PBF) was constructed externally using Osmosis, due to R's limitations
in handling PBF file merging. This composite road network served as the
base layer for accessibility analysis. Additionally, the most recent
General Transit Feed Specification (GTFS, 2024) data for Mexico City
were obtained from its open data portal.

Multimodal travel time matrices were computed using the r5r routing
engine. For each origin--destination pair, travel times were estimated
under two transportation scenarios: private motor vehicle and public
transit (combined with walking). The routing represented conditions on a
typical weekday at 8:00 a.m. It is important to consider that r5r
computed private travel times do not account for traffic.

Accessibility was then assessed for seven types of opportunities
relevant to residents: the number of jobs, and the number of economic
units in sectors such as retail, education, health, recreation, and
gastronomy---based on 2-digit NAICS codes (and 3-digit for gastronomy,
only)---as well as public spaces as defined by the CDMX government. For
each category, two accessibility measures were calculated: a cumulative
opportunity measure, which counts the number of destinations reachable
within 10, 20, and 30 minutes, and a gravity-based measure, which
weights opportunities by their proximity using an exponential decay
function with multiple parameters (0.05, 0.1, and 0.2). Accessibility
was calculated under both public transit and private vehicle scenarios,
resulting in a total of 28 indicators for each transport mode,
destination type, and analytical method.

\section{NEIGHBOURHOOD
CHARACTERISTICS}\label{neighbourhood-characteristics}

To contextualize real estate listings within neighborhood conditions,
both crime data and block-level social development indicators were
incorporated. The crime dataset included georeferenced events from 2023
onward, published on the Mexico City open data portal, focusing only on
incidents with physical harm. Each listing was spatially joined to
nearby crime occurrences within a 500-meter buffer to estimate localized
exposure to violent events. In addition, listings were linked to the
corresponding block-level Social Development Index, a composite metric
ranging from 0 to 1. This index represents an average of multiple
dimensions of deprivation, including housing quality, sanitation, energy
access, telecommunications, essential goods availability, education,
social security, and health services. To further account for contextual
variation, the corresponding municipality of each listing was registered
as well to include heterogeneity across administrative boundaries.

\section{REGRESSION SPECIFICATION}\label{regression-specification}

The hedonic price model was specified as a log-log regression, estimated
separately for both private and public accessibility, according to the
following structure:

\[
log(Price_i) = \alpha + \beta log(Accessibility_{i}) + \gamma Dwelling_{i} + \delta Neighbourhood_{i} + \mu_i
\]

where:

\begin{itemize}
\tightlist
\item
  \(i\) represents individual real estate listings;
\item
  \(Price_i\) denotes the listing price in USD for property \(i\);
\item
  \(Accessibility_{i}\) is a vector of accessibility variables for
  listing \(i\) corresponding to different destination types (jobs,
  retail, education, health, recreation, gastronomy, and public spaces);
\item
  \(Dwelling_{i}\) is a vector of dwelling characteristics for listing
  \(i\), including constructed area (in squared meters), number of
  bedrooms, bathrooms, and parking spaces;
\item
  \(Neighbourhood_{i}\) is a vector of neighbourhood-level
  characteristics of listing \(i\), including the number of reported
  crimes within a 500m radius, the Social Development Index of the
  corresponding census block, and a categorical fixed effect for the
  listing's municipality, capturing administrative-level variation.
\item
  \(\mu_{i}\) is the error term.
\end{itemize}

To evaluate model validity and ensure robust inference, standard
regression diagnostics were conducted. The Breusch-Pagan test indicated
the presence of heteroskedasticity, while the Durbin-Watson test
suggested slight positive autocorrelation in the residuals. Furthermore,
multicollinearity was identified among some accessibility indicators
according to the Variance Inflation Factor (VIF) diagnostics---an
expected outcome given the conceptual overlap between destination types.
However, rather than removing these variables or applying dimensionality
reduction techniques such as principal component analysis (PCA), all
indicators were retained, as they represent the core constructs under
study.

While these violations do not bias the estimated coefficients themselves
under OLS assumptions, they can affect the reliability of standard
errors and inference. To address these issues while preserving the
integrity of the specification, robust standard errors (via the
Huber-White estimator) and municipality-clustered standard errors were
employed. This approach corrects for heteroskedasticity and allows for
arbitrary correlation of errors within municipalities, thereby
addressing both multicollinearity-related inference issues and potential
spatial autocorrelation within administrative boundaries. Residual
normality was also assessed for both models using the Shapiro-Wilk test,
which indicated deviations from normality. Given the moderate sample
size and the use of robust and clustered standard errors, inference can
be considered reliable despite this violation.

\newpage

\chapter{RESULTS}\label{results}

This section presents the main empirical findings of the study. It is
organized into three parts. First, we analyze the construction and
behavior of the accessibility index, examining its methodological
structure, sector-specific patterns, and spatial distribution. Second,
we explore the dynamics of the real estate market, including overall
price distributions and their relationship to accessibility levels.
Finally, we present the results of hedonic price models that quantify
the influence of accessibility and other factors on real estate values.

\section{ACCESSIBILITY INDEX}\label{accessibility-index-1}

This subsection presents a comprehensive accessibility analysis for
Mexico City (CDMX), using urban blocks---the most detailed spatial
resolution available from national census data---as the unit of origin.
This approach captures how accessibility to jobs and essential services
varies across the city, extending beyond the scope of real estate
listings to include the broader urban structure.

\subsection{METHODS AND INTERVAL
ANALYSIS}\label{methods-and-interval-analysis}

Figures 1 and 2 compare the distribution of job accessibility indices
across Mexico City in 2024, using two measurement
methodologies---cumulative and gravity-based---for both public and
private modes of transportation.

Figure 1 illustrates public job accessibility, accounting for walking
and public transport as the primary modes of travel. The distribution of
accessibility values is heavily right-skewed under both methodologies,
with most spatial units exhibiting relatively low access to public
sector employment under high impedance scenarios. As impedance
decreases---either by extending time thresholds or reducing the decay
parameter---the accessibility index increases, resulting in a broader
and more dispersed distribution.

\begin{figure}[H]

{\centering \includegraphics{output_vis/acc_methods_public_hist.png}

}

\caption{Public Job Accessibility by Measurement Methodology}

\end{figure}%

In contrast, Figure 2 presents private job accessibility, estimated
using car-based travel. Across all impedance levels, distributions show
substantially higher accessibility values, reflecting the broader
spatial distribution of private employment and the relative advantage of
car access in reaching jobs. In the cumulative panel, accessibility
increases markedly as travel time thresholds expand, shifting the
frequency distribution rightward from high to low impedance. This shift
suggests that, under 20- or 30-minute thresholds, most job destinations
become accessible when traffic is not considered---potentially
diminishing the interpretive value of cumulative measures under such
assumptions. The gravity-based panel exhibits decreasing concentration
with increasing impedance, consistent with the logic of exponential
decay applied in its computation.

\begin{figure}[H]

{\centering \includegraphics{output_vis/acc_methods_private_hist.png}

}

\caption{Private Job Accessibility by Measurement Methodology}

\end{figure}%

Comparing the two figures highlights clear disparities between
transportation modes. Private job accessibility not only yields higher
values but also displays greater variance than public job accessibility.
This suggests that individuals dependent on walking and public transport
face more constrained access to employment opportunities. Furthermore,
the interpretive validity of cumulative private accessibility under low
impedance may be limited in contexts where congestion and traffic
dynamics are excluded from travel time estimates.

Figures 3 and 4 explore the degree of concordance between cumulative and
gravity-based accessibility measures across various sectors in Mexico
City during 2024. Based on the interpretations of Figures 1 and 2, the
remaining of the present study focuses on low-impedance public
accessibility and high-impedance private accessibility. Across both
figures, a strong linear relationship is evident between the two
measurement methodologies.

In the public accessibility context (Figure 3), all sectors exhibit
near-perfect correlations (R ≥ 0.99), with employment, education,
recreation, and gastronomy showing virtual correlations of R = 1. This
indicates that under low-impedance public transport conditions,
cumulative and gravity-based measures yield highly consistent results,
suggesting methodological interchangeability when evaluating access to
these types of destinations.

\begin{figure}[H]

{\centering \includegraphics{output_vis/acc_methods_public_aggr_corr.png}

}

\caption{Public Accessibility by Sector and Measurement Technology}

\end{figure}%

In the private accessibility context (Figure 4), correlations remain
high but are slightly lower than in the public case, ranging from R =
0.96 (health, public spaces) to R = 0.98 (employment, education,
gastronomy). Although the distributions still align closely with the 1:1
line, greater dispersion is observed relative to the public
accessibility plots. This suggests a somewhat higher sensitivity to
methodological differences under high-impedance conditions; nonetheless,
the results remain strongly aligned.

\begin{figure}[H]

{\centering \includegraphics{output_vis/acc_methods_private_aggr_corr.png}

}

\caption{Private Accessibility by Sector and Measurement Technology}

\end{figure}%

Overall, the comparison highlights a notable degree of alignment between
cumulative and gravity-based methodologies across both sectors and
travel modes. While gravity-based methods are often favored for their
ability to model distance decay effects, cumulative measures may be
preferred for their simplicity and interpretability. In this context,
either method appears to serve as a valid proxy for the other under
appropriate impedance assumptions. These findings reinforce the internal
consistency of the accessibility modeling framework and provide
methodological flexibility for practitioners choosing between ease of
implementation and behavioral precision in accessibility assessments.
For simplicity and easer interpretation, the remaining of this study
will consider cumulative metrics.

\subsection{SECTORIAL ANALYSIS}\label{sectorial-analysis}

Figures 5 through 8 display the distribution of sectoral accessibility
values in Mexico City during 2024 using violin plots, distinguishing
between public and private accessibility and further separating
employment-based (Figures 5 and 6) and economic unit-based (Figures 7
and 8) indicators. These plots provide a more granular view of
accessibility by economic sector, while also revealing the skewness and
concentration of opportunities across Mexico City.

Figure 5 presents public accessibility to employment. The distributions
are notably right-skewed, particularly for \emph{Professional \&
Administrative} and \emph{Industrial \& Logistics} sectors, where a
large number of spatial units cluster at low accessibility levels.
\emph{Retail \& Services} shows a relatively flatter and broader
distribution, indicating more spatial dispersion in access to these
jobs.

\begin{figure}[H]

{\centering \includegraphics{output_vis/acc_sectorial_public_employment_violin.png}

}

\caption{Public Employment Accessibility Density by Sector}

\end{figure}%

In contrast, Figure 6 shows private accessibility to employment. All
three employment categories display significantly higher accessibility
values compared to their public counterparts, with distributions
stretching further to the right. Nevertheless, strong concentration
remains at lower access values, especially in Professional \&
Administrative and Industrial \& Logistics sectors, highlighting
persistent spatial disparities even in a car-oriented access model.

\begin{figure}[H]

{\centering \includegraphics{output_vis/acc_sectorial_private_employment_violin.png}

}

\caption{Private Employment Accessibility Density by Sector}

\end{figure}%

Figure 7, representing public accessibility, again exhibits strong
right-skewed distributions across all sectors. The most constrained
access is seen in Recreation, Public Spaces, and Education, where a
large share of the population has access to a limited number of
facilities.

\begin{figure}[H]

{\centering \includegraphics{output_vis/acc_sectorial_public_units_violin.png}

}

\caption{Public Economic Accessibility Density by Sector}

\end{figure}%

In Figure 8, private accessibility shows broader distributions and
higher accessibility values across all sectors. The accessibility to
Retail and Gastronomy units is particularly elevated, likely due to
their dense and dispersed urban presence. However, Recreation and Public
Spaces remain skewed toward lower values, suggesting that even with
private mobility, access to urban amenities of collective use remains
relatively geographically limited.

\begin{figure}[H]

{\centering \includegraphics{output_vis/acc_sectorial_private_units_violin.png}

}

\caption{Private Economic Accessibility Density by Sector}

\end{figure}%

\subsection{SPATIAL ANALYSIS}\label{spatial-analysis}

Figures 9 and 10 present the spatial distribution of average
accessibility z-scores for public and private modes, respectively,
aggregated across all analyzed sectors. The values represent
standardized accessibility per urban block, with blue areas denoting
above-average access and red areas reflecting below-average access.

In Figure 9, public accessibility reveals a pronounced core-periphery
pattern, aligned with public transit infrastructure and routes. Central
blocks exhibit consistently high accessibility across sectors, forming a
dense cluster of blue and purple zones. In contrast, peripheral zones
are marked by deep red, indicating significant deviations below the
citywide mean. This highlights stark geographical disparities in access
to basic services and economic opportunities for transit-dependent
populations.

\begin{figure}[H]

{\centering \includegraphics{output_vis/acc_public_avg_z_map.png}

}

\caption{Average Public Accessibility per Urban Block}

\end{figure}%

Figure 10, which maps private accessibility displays a similar spatial
trend, but with noticeably more diffuse gradients. While central zones
still dominate in terms of accessibility, a broader spread of light-blue
and neutral zones suggests that private vehicles slightly attenuate the
intensity of spatial inequality, allowing some peripheral areas to reach
modest accessibility levels. However, deep-red zones remain widespread
in the urban periphery, especially to the south.

\begin{figure}[H]

{\centering \includegraphics{output_vis/acc_private_avg_z_map.png}

}

\caption{Average Private Accessibility per Urban Block}

\end{figure}%

Figures 11 and 12 disaggregate these results by \textbf{sector},
offering a more detailed visualization of public (Figure 11) and private
(Figure 12) accessibility for each service type per urban block.

\begin{figure}[H]

{\centering \includegraphics{output_vis/acc_public_aggr_z_map.png}

}

\caption{Public Accessibility by Sector per Urban Block}

\end{figure}%%
\begin{figure}[H]

{\centering \includegraphics{output_vis/acc_private_aggr_z_map.png}

}

\caption{Private Accessibility by Sector per Urban Block}

\end{figure}%

A comparison of the two set of maps reaffirms the broader reach of
private transportation, with blue zones extending farther from the city
center across all sectors. Yet even with car access, spatial inequality
persists. The z-score standardization facilitates comparability across
sectors, making it evident that accessibility is not uniformly
distributed, regardless of transport mode. Together, these figures
underscore a persistent spatial divide in Mexico City's accessibility
landscape. While private vehicles improve reach, they do not eliminate
systemic disparities. Conversely, public transit accessibility remains
highly centralized.

\section{REAL ESTATE MARKET}\label{real-estate-market}

This section examines the structure of the real estate market in Mexico
City through the lens of property listings, focusing on both price
dynamics and spatial accessibility. By analyzing listing data
disaggregated by housing typology, we explore how prices vary across the
city in both absolute terms and per square meter. These patterns are
then contextualized within an accessibility framework, assessing how
well-connected each listing is.

\subsection{PRICE DISTRIBUTION}\label{price-distribution}

Figures 13 and 14 compare the distribution of property values derived
from real estate listings in Mexico City, disaggregated by houses and
flats to contextualize within prevailing market dynamics.

Figure 13 illustrates the distribution of absolute prices. Houses
dominate the lower-to-mid price range, with a sharp peak between 100K
and 300K USD, suggesting a concentration in more affordable areas.
Flats, by contrast, exhibit a flatter and more extended distribution,
with a notable presence in higher price brackets exceeding 500K USD.

\begin{figure}[H]

{\centering \includegraphics{output_vis/re_price_hist.png}

}

\caption{Dwelling Price Distribution by Type}

\end{figure}%

Figure 14 shifts focus to price per square meter, revealing a more
nuanced distinction between dwelling types. Flats tend to cluster in the
lower unit price ranges, with most listings concentrated between 1K and
2K USD per square meter, suggesting a higher frequency of smaller or
more standardized units in relatively lower-cost locations. Houses, by
contrast, display a broader distribution skewed toward higher
price-per-square-meter values, peaking between 2K and 3K USD and
extending further into the upper range.

\begin{figure}[H]

{\centering \includegraphics{output_vis/re_price_m2_hist.png}

}

\caption{Dwelling Price per m2 Distribution per Type}

\end{figure}%

Figure 15 shows the spatial distribution of real estate listings used in
the sample, represented as point data across Mexico City (CDMX). The
spatial pattern reveals a strong central clustering, with the highest
concentration of listings located in and around the urban core,
particularly in the central-western zones. While the density of listings
decreases toward the urban periphery, the dataset achieves broad spatial
coverage across most of Mexico City's territory.

\begin{figure}[H]

{\centering \includegraphics{output_vis/re_sample_map.png}

}

\caption{Real Estate Listings}

\end{figure}%

\subsection{ACCESSIBILITY ANALYSIS}\label{accessibility-analysis}

Figure 16 displays the relationship between real estate listing prices
(per square meter) and public transport accessibility. Results show that
45.8\% of listings fall into the low-access, low-price quadrant. A
smaller share (18.3\%) appears in the low-access, high-price quadrant.
Listings with both high accessibility and high prices account for only
22.9\%, while the high-access, low-price quadrant---a segment of
particular interest for affordability and value---represents just 12.9\%
of the total.

\begin{figure}[H]

{\centering \includegraphics{output_vis/re_acc_pub_scat.png}

}

\caption{Real Estate Prices vs.~Average Public Accessibility}

\end{figure}%

The spatial dynamics are further illustrated in Figure 17, which maps
the quadrant classifications from the public accessibility analysis. The
low-access, low-price listings dominate the urban periphery. The
high-access, high-price properties cluster around the city core and
well-connected central-western corridors. Only a few high-access,
low-price listings are visible, while low-access, high-price listings
are scattered across less connected but still expensive zones.

\begin{figure}[H]

{\centering \includegraphics{output_vis/re_acc_pub_map.png}

}

\caption{Real Estate Listings by Public Price-Access Quadrant}

\end{figure}%

Figure 18 plots the same relationship using private accessibility. The
share of listings in the high-access, high-price quadrant increases to
26.5\%. Similarly, the high-access, low-price segment grows to 20.6\%.
Meanwhile, the low-access, low-price quadrant decreases to 38.2\%, and
the low-access, high-price quadrant drops to 14.8\%. Overall, private
accessibility appears to reduce the concentration of listings in the
most disadvantaged category while expanding the potential for spatial
opportunity in areas where public transport remains limited.

\begin{figure}[H]

{\centering \includegraphics{output_vis/re_acc_priv_scat.png}

}

\caption{Real Estate Prices vs.~Average Private Accessibility}

\end{figure}%

Figure 19, based on private accessibility, shows a notable spatial
expansion of the high-access quadrants. High-access/high-price listings
extend further beyond the core into more peripheral but car-connected
areas, while affordable properties with high private accessibility are
more broadly distributed across intermediate zones. The
low-access/low-price area contracts, though it remains dominant in the
far periphery, and the low-access/high-price is less prominent.

\begin{figure}[H]

{\centering \includegraphics{output_vis/re_acc_priv_map.png}

}

\caption{Real Estate Listings by Private Price-Access Quadrant}

\end{figure}%

\section{HEDONIC PRICE MODEL RESULTS}\label{hedonic-price-model-results}

This section presents the results of the hedonic pricing models
examining the relationship between real estate listing prices and
accessibility, estimated separately for public and private transport
modes. Both models include accessibility to seven destination
types---employment, education, gastronomy, health, public spaces,
recreation, and retail---while controlling for structural dwelling
attributes and neighborhood characteristics, including local crime
rates, a social development index, and municipality-level fixed effects.
The log-log specification allows coefficients to be interpreted as
elasticities.

\subsection{PUBLIC ACCESSIBILITY}\label{public-accessibility}

The results for the public-based model are presented in Figure 20. The
model explains a high proportion of the variation in housing prices,
with an adjusted R² of 0.837 across all specifications. The most robust
and statistically significant predictor is employment accessibility,
which has a positive elasticity of 0.242 (p \textless{} 0.001, for the
three scenarios), with a 95\% confidence interval of {[}0.127, 0.357{]}
under the most conservative scenario with clustered standard errors.
This suggests that---caeteris paribus and on average--- a 10\% increase
in cumulative job accessibility is associated with approximately a 2.4\%
increase in listing price.

Education accessibility is negatively associated with price, with a
coefficient of --0.155. This effect is statistically significant under
OLS (CI: {[}--0.266, --0.044{]}) and robust standard errors (CI:
{[}--0.281, --0.029{]}), indicating that greater proximity to
educational facilities is caeteris paribus, on average, related with
lower listing prices. However, the effect loses statistical significance
under clustered standard errors (CI: {[}--0.383, 0.073{]}).

Gastronomy accessibility has a positive effect on housing prices, with a
coefficient of 0.169 that is statistically significant under both OLS
and robust standard error specifications (p \textless{} 0.05), and a
95\% confidence interval of {[}0.006, 0.331{]} in the latter. However,
when clustering standard errors at the municipal level, the confidence
interval widens substantially to {[}--0.113, 0.450{]}, encompassing zero
and indicating that the effect may not be robust to spatial correlation.

A similar pattern is observed for health accessibility, which is
positively associated with price in the OLS and robust models with a
coefficient of 0.066 (p \textless{} 0.05), and a 95\% CI of {[}0.005,
0.127{]} under both. Yet, the effect becomes statistically insignificant
when clustering is applied (CI: {[}--0.055, 0.187{]}), suggesting that
these associations may not be robust to spatial correlation.

In contrast, public space and recreation accessibility show no
significant relationship with listing prices under any model
specification, with confidence intervals consistently centered around
zero, indicating a lack of systematic association.

Notably, retail accessibility exhibits a consistently negative effect
across all estimation strategies, with a coefficient of --0.274 and a
95\% confidence interval under clustered standard errors of {[}--0.441,
--0.107{]}. This suggests that---caeteris paribus and on average---a
10\% increase in cumulative retail accessibility is associated with
approximately a 2.7\% decrease in listing price. This robust finding
suggests that, contrary to conventional expectations, proximity to
retail may be perceived as a disamenity.

Taken together, these results highlight the differentiated role of
public accessibility dimensions in shaping housing prices. While
employment accessibility emerges as the most robust and consistently
positive determinant, other amenities such as education, health, and
gastronomy exhibit more context-dependent effects that weaken once
spatial clustering is accounted for. The lack of association for public
space and recreation suggests these amenities may not be capitalized
into housing prices in a systematic way, at least not through the
mechanisms captured here. Conversely, the negative capitalization of
retail accessibility---persistent across all model
specifications---challenges common assumptions about commercial
proximity as a universal benefit.

\begin{figure}[H]

{\centering \includegraphics{output_vis/regression_table_public.png}

}

\caption{Public Accessibility Model (Access Variables Only)}

\end{figure}%

\subsection{PRIVATE ACCESSIBILITY}\label{private-accessibility}

The results for the private-based model are presented in Figure 21. The
model demonstrates strong explanatory power, with an adjusted R² of
0.839 across all specifications, indicating that the model accounts for
a large proportion of the variation in listing prices. As in the public
model, the most consistent and robust predictor is employment
accessibility, which has a positive elasticity of 0.540 (p \textless{}
0.001 under OLS and robust SEs; p \textless{} 0.05 under clustered SEs).
The 95\% confidence interval under clustered standard errors is
{[}0.116, 0.963{]}, reinforcing the interpretation that---caeteris
paribus and on average---a 10\% increase in car-based employment
accessibility is associated with a 5.4\% increase in listing price. This
effect is not only statistically significant but also substantively
stronger than in the public accessibility model.

Education accessibility is negatively associated with price, with a
coefficient of --0.209. The effect is statistically significant in the
OLS model (p \textless{} 0.05, CI: {[}--0.402, --0.015{]}) and
marginally significant under robust standard errors (p \textless{} 0.1,
CI: {[}--0.422, 0.004{]}), but the effect weakens under clustered
standard errors, where the 95\% confidence interval expands to
{[}--0.675, 0.258{]} and crosses zero.

For other amenities, gastronomy accessibility shows no statistically
significant association with price under any specification. Its
coefficient is positive but close to zero (0.044), and the clustered
confidence interval ({[}--0.421, 0.510{]}) confirms the absence of a
reliable relationship. Similarly, health accessibility has no effect on
prices, with a coefficient of --0.006 and a clustered CI of {[}--0.232,
0.220{]}.

Public spaces and recreation accessibility likewise do not show
statistically significant effects. Public spaces have a positive
coefficient of 0.055, but the clustered confidence interval ({[}--0.067,
0.177{]}) crosses zero. Recreation has a negative coefficient (--0.065)
and similarly wide and inconclusive confidence intervals {[}--0.440,
0.310{]}, indicating that neither amenity is systematically capitalized
into listing prices when measured through private access.

As in the public accessibility model, retail accessibility exhibits a
consistently strong and negative association with listing price. The
coefficient is --0.326, statistically significant at p \textless{} 0.001
across all specifications. The clustered confidence interval
({[}--0.496, --0.157{]}) remains narrow and well below zero, confirming
the robustness of this finding. This suggests that---caeteris paribus
and on average---a 10\% increase in cumulative car-based retail
accessibility is associated with a 3.3\% decrease in listing price.

These results highlight the differentiated influence of private
accessibility dimensions on housing prices in Mexico City. Employment
accessibility by car stands out as the most robust and consistently
positive predictor, suggesting that job reachability via private
transport is highly valued and strongly capitalized into property
values. In contrast, amenities such as education, health, gastronomy,
public spaces, and recreation show limited or inconsistent effects, with
their associations weakening or disappearing once spatial dependence is
accounted for. As in the public model, the consistently negative
relationship between retail accessibility and price challenges
assumptions about the inherent desirability of commercial proximity,
suggesting that higher retail density may, in some contexts, signal
disamenities that detract from residential value.

\begin{figure}[H]

{\centering \includegraphics{output_vis/regression_table_private.png}

}

\caption{Private Accessibility Model (Access Variables Only)}

\end{figure}%

\newpage

\chapter{DISCUSSION}\label{discussion}

The findings of this study offer important insights into how various
forms of accessibility are capitalized into housing prices in Mexico
City. Drawing from hedonic pricing models, the results highlight spatial
disparities in access to employment and amenities, while also revealing
the differentiated influence of accessibility across transportation
modes and destination types. These patterns contribute to ongoing
debates in the accessibility literature regarding how access is
measured, perceived, and valued in real estate markets (Boisjoly \&
El-Geneidy, 2017; Levinson \& Wu, 2020; Heyman et al., 2018).

The accessibility analysis confirms the persistence of a pronounced
core--periphery divide in Mexico City, especially under public transport
conditions. This pattern mirrors findings by Atuesta et al.~(2018), who
observed that employment accessibility in Mexico City remains highly
concentrated in central areas and unequally distributed along
socioeconomic lines. In particular, high-income households tend to
benefit from proximity to job centers and mobility via private vehicles,
while low-income populations are often relegated to peripheral zones
with limited access to formal employment and public transport. The
spatial clustering of high public accessibility in central and western
districts underscores these structural disparities, reinforcing the
barriers faced by transit-dependent populations in outlying areas.
Although private vehicle access slightly reduces this divide by
extending reach into intermediate zones, it does not overcome the
underlying inequality in spatial opportunity.

Building on these foundational insights, the present study advances the
analysis of spatial accessibility by disaggregating access by
sector---an innovation that reveals important nuances in how different
types of opportunities are distributed across the city. While previous
studies have primarily focused on employment, this analysis incorporates
a diverse range of destinations, including education, health, retail,
gastronomy, recreation, and public spaces. This multidimensional
approach uncovers uneven spatial patterns across sectors, suggesting
that access to certain amenities remains highly centralized or limited
depending on the mode of transport.

Methodologically, the study also addresses several limitations common in
urban accessibility research. First, by using highly disaggregated
spatial units---urban blocks as origins---it helps mitigate the
Modifiable Areal Unit Problem (MAUP), which can distort results when
coarser or arbitrary zones are used for modeling (Heyman et al., 2018).
Second, accessibility is computed using both cumulative and
gravity-based approaches, capturing different dimensions of spatial
reach. The strong correlation between both measures suggests that
simpler metrics may serve as practical proxies for more complex
formulations under certain conditions. This supports the claim by Siddiq
and Taylor (2021) that cumulative opportunities can be effective tools
in planning practice, especially when resources for behavioral modeling
are limited. However, the dispersion observed in the private mode
underscores that sensitivity to measurement assumptions increases with
impedance, suggesting that planners should exercise caution when
interpreting accessibility under longer travel time assumptions or in
uncongested car-centric models.

Moreover, when accessibility is viewed in relation to housing values,
the results from the quadrant analysis show that low accessibility does
not always equate to lower property prices. In fact, some high-value
residential areas exhibit poor accessibility levels, suggesting that
spatial isolation may be perceived as a desirable feature by
higher-income households seeking privacy, exclusivity, or distance from
urban intensity. These dynamics echo Wachs and Kumagai's (1973)
conception of physical accessibility as a social indicator---where
access not only reflects urban structure but also reveals embedded
patterns of exclusion shaped by income, infrastructure, and urban form.

When accessibility is linked to property values through hedonic
modeling, the findings affirm Rosen's (1974) theory that spatial
attributes---such as reachability of destinations---are capitalized into
housing prices as part of the location bundle. Employment accessibility
emerges as the most robust predictor across both models, with
elasticities ranging from 0.24 under public transit to 0.54 under
car-based access. This reinforces the central role of job accessibility
in shaping urban land values, consistent with evidence from developing
contexts (Atuesta et al., 2018).

Compared to Atuesta et al.'s (2018) estimates---suggesting that a 1\%
reduction in distance to employment subcenters increases housing values
by 1--3\%---the elasticities reported here are notably smaller in
relative terms. This difference partly reflects methodological
divergences: whereas Atuesta et al.~rely on distance to predefined
employment subcenters as a proxy for accessibility, the present study
adopts a cumulative opportunities approach, computed from census
block-level origins to thousands of georreferenced destinations,
providing a more spatially granular representation of access. Moreover,
the regression framework employed here incorporates a broader set of
control variables, including dwelling characteristics, local crime
exposure, social development indicators, and municipal fixed
effects---enhancing the robustness of the estimates by reducing
potential omitted variable bias. Nevertheless, differences in data
sources and time periods likely contribute as well. Atuesta's analysis
covers standardized housing projects from 2002 to 2013, whereas this
study examines real estate listings from 2024, capturing a more
heterogeneous housing market under contemporary spatial and
institutional conditions. Taken together, these factors help explain why
the magnitude of capitalization differs, while affirming the consistent
directional importance of employment accessibility in urban housing
markets.

The present study also challenges conventional assumptions about the
uniform value of accessibility. Access to education, health, gastronomy,
and public spaces shows limited or context-dependent effects, with many
associations weakening once spatial clustering is accounted for. Perhaps
most striking is the consistent negative association between retail
accessibility and housing prices---across both transport modes and all
specifications. This challenges widely held assumptions about linear
relationships between accessibility and house prices, potentially
exhibiting thresholds where its effect saturates or diminishes (e.g.,
Liu et al., 2024) suggesting that retail density may also signal
congestion, noise, or informal activity, especially in large cities.

It is important to acknowledge, however, that the accessibility
indicators used in this study are based on broad 2-digit classifications
of economic units following the NAICS system, without further
disaggregation by quality, specialization, or service intensity. As a
result, these categories may encompass a wide variety of establishments
with differing levels of desirability, functionality, or user
perception. This aggregation could obscure localized effects, especially
in cases where accessibility to higher-quality or better-managed
facilities might positively influence property values, while proximity
to lower-quality or oversaturated services might do the opposite. In
addition, the current structure does not account for the size or
capacity of each economic unit. All establishments are weighted equally,
regardless of their employment level. While alternative
approaches---such as measuring accessibility by total jobs per
sector---could provide more nuanced estimates of economic scale and
influence, they also introduce new complications. Specifically, job
counts may reflect administrative or support roles unrelated to the
primary function of the establishment, or incorporate sectoral
distortions due to firm size and internal hierarchies, thereby
complicating the behavioral interpretation of access. These limitations
align with Heyman et al.'s (2018) critique that proximity alone cannot
fully capture the experiential or qualitative dimensions of
accessibility, particularly in dense urban contexts where perceptions of
congestion, noise, or informal activity may mediate the value of nearby
amenities.

An additional methodological consideration concerns the presence of
multicollinearity among the accessibility indicators. Because many
destination types tend to co-locate in urban space, their respective
accessibility measures are often highly correlated. While
multicollinearity does not bias coefficient estimates, it can inflate
standard errors and thereby reduce the statistical significance of
individual predictors, potentially obscuring meaningful relationships.
This is particularly relevant when interpreting the weaker or
inconsistent effects observed for certain amenities. In this study, all
accessibility indicators were deliberately retained to preserve the
sectoral focus and interpretive clarity of the model, rather than apply
dimensionality reduction techniques like PCA. To mitigate the
inferential uncertainty that may result from multicollinearity---as well
as spatial autocorrelation---clustered standard errors at the municipal
level were employed. This specification produces more conservative
estimates, allowing for correlated error structures within local
submarkets while maintaining the integrity of the model's structural
assumptions.

Moreover, while the model controls for a range of structural and
neighborhood-level factors and incorporates fixed effects and clustered
standard errors at the municipal level, it is important to recognize
that the analysis is designed to capture broader, citywide patterns
rather than localized dynamics. As such, the model is more reflective of
average relationships across the urban region than of specific
neighborhoods or submarkets. These local dynamics, shaped by complex
socio-spatial histories, land use patterns, and community preferences,
may exert significant influence on housing prices but remain unobserved
in this framework. While neighborhood-level controls such as crime
incidence and social development index help account for some of this
variation, the absence of finer-scale fixed effects and the sample
distribution reflects a necessary tradeoff driven by data and sample
size constraints. As such, the findings should be interpreted as
identifying average structural relationships across Mexico City, rather
than providing full explanatory coverage of localized valuation
processes.

These results highlight a tension between accessibility as a normative
planning goal and accessibility as a market-valued trait. While policy
frameworks often prioritize increasing access to services and jobs for
equity and efficiency reasons (El-Geneidy \& Levinson, 2022), the
housing market may assign variable and sometimes negative value to these
same features, depending on how they intersect with perceptions of
quality, safety, and environmental context. This underscores the
importance of integrating behavioral realism and local contextual
sensitivity into accessibility modeling---especially when applying these
insights to housing policy, transit-oriented development, or land value
capture mechanisms.

\newpage

\chapter{CONCLUSION}\label{conclusion}

XXX

\newpage

\chapter{REFERENCES}\label{references}

\begin{itemize}
\item
  Atuesta, L. H., Ibarra-Olivo, J. E., Lozano-Gracia, N., \& Deichmann,
  U. (2018). Access to employment and property values in Mexico.
  \emph{Regional Science and Urban Economics}, \emph{70}, 142--154.
  https://doi.org/10.1016/j.regsciurbeco.2018.03.005
\item
  Boisjoly, G., \& El-Geneidy, A. M. (2017). The insider: A planners'
  perspective on accessibility. \emph{Journal of Transport Geography},
  \emph{64}, 33--43.
\item
  Cervero, R., Rood, T., \& Appleyard, B. (1999). Tracking
  accessibility: Employment and housing opportunities in the San
  Francisco Bay Area. \emph{Environment and Planning A: Economy and
  Space}, \emph{31}(7), 1259--1278. https://doi.org/10.1068/a311259
\item
  El-Geneidy, A., \& Levinson, D. (2022). Making accessibility work in
  practice. \emph{Transport Reviews}, \emph{42}(2), 129--133.
\item
  Evans, A. W. (1995). The property market: Ninety per cent efficient?
  \emph{Urban Studies}, \emph{32}(1), 5--29.
  https://doi.org/10.1080/00420989550013194
\item
  Hansen, W. G. (1959). How accessibility shapes land use. \emph{Journal
  of the American Institute of Planners}, \emph{25}(2), 73--76.
  https://doi.org/10.1080/01944365908978307
\item
  Heyman, A. V., Law, S., \& Berghauser Pont, M. (2018). How is location
  measured in housing valuation? A systematic review of accessibility
  specifications in hedonic price models. \emph{Urban Science},
  \emph{3}(1), 3. https://doi.org/10.3390/urbansci3010003
\item
  Khoshnoud, M., Sirmans, G. S., \& Zietz, E. N. (2023). The evolution
  of hedonic pricing models. \emph{Journal of Real Estate Literature},
  \emph{31}(1), 1--47. https://doi.org/10.1080/09277544.2023.2201020
\item
  Levinson, D. M., \& Wu, H. (2020). Towards a general theory of access.
  \emph{Journal of Transport and Land Use}, \emph{13}(1), 129--158.
\item
  Liu, X., Chen, X., Orford, S., Tian, M., \& Zou, G. (2024). Does
  better accessibility always mean higher house prices?
  \emph{Environment and Planning B: Urban Analytics and City Science},
  \emph{51}(9), 2179--2195. https://doi.org/10.1177/23998083241242212
\item
  Rosen, S. (1974). Hedonic prices and implicit markets: Product
  differentiation in pure competition. \emph{Journal of Political
  Economy}, \emph{82}(1), 34. https://doi.org/10.1086/260169
\item
  Siddiq, F., \& Taylor, B. D. (2021). Tools of the trade?: Assessing
  the progress of accessibility measures for planning practice.
  \emph{Journal of the American Planning Association}, \emph{87}(4),
  497--511.
\item
  Wallace, N. E., \& Meese, R. A. (1997). The construction of
  residential housing price indices: A comparison of repeat-sales,
  hedonic-regression, and hybrid approaches. \emph{The Journal of Real
  Estate Finance and Economics}, \emph{14}(1/2), 51--73.
  https://doi.org/10.1023/A:1007715917198
\item
  Wachs, M., \& Kumagai, T. G. (1973). Physical accessibility as a
  social indicator. \emph{Socio-Economic Planning Sciences},
  \emph{7}(5), 437--456.
\item
  Wei, C., Fu, M., Wang, L., Yang, H., Tang, F., \& Xiong, Y. (2022).
  The research development of hedonic price model-based real estate
  appraisal in the era of big data. \emph{Land}, \emph{11}(3), 334.
  https://doi.org/10.3390/land11030334
\end{itemize}




\end{document}
